%%%%%%%%%%%%%%%%%%%%%%%%%%%%%%%%%%%%%%%% IMPORTS %%%%%%%%%%%%%%%%%%%%%%%%%%%%%%%%%%%%%%%%
\documentclass[10pt,onesize,a4paper,titlepage]{article}

%%%%%%%%%%%%%%% Formatting %%%%%%%%%%%%%%% 
\usepackage[english]{babel}
\usepackage[utf8]{inputenc}
\usepackage{geometry} % Margins
\usepackage{sectsty} % Custom Sections

%%%%%%%%%%%%%%% Font %%%%%%%%%%%%%%% 
\usepackage{Archivo}
\usepackage[T1]{fontenc}

%%%%%%%%%%%%%%% Graphics %%%%%%%%%%%%%%% 
\usepackage{fontawesome5} % Icons
\usepackage{graphicx} % Images
\usepackage[most]{tcolorbox} % Color Box
\usepackage{xcolor} % Colors
\usepackage{tikz} % For Drawing Shapes
\tcbuselibrary{breakable}

%%%%%%%%%%%%%%% Miscelanous %%%%%%%%%%%%%%% 
\usepackage{lipsum} % Lorem Ipsum
\usepackage{hyperref} % For Hyperlinks

%%%%%%%%%%%%%%% Miscelanous %%%%%%%%%%%%%%% 
\graphicspath{{../pictures/}}

%%%%%%%%%%%%%%% Colors %%%%%%%%%%%%%%% 
\definecolor{title}{HTML}{4bfbba} % Color of the title
\definecolor{backdrop}{HTML}{f2f2f2} % Color of the side column
\definecolor{lightgray}{HTML}{b8b8b8} % Color for the skill bars

%%%%%%%%%%%%%%% Section Format %%%%%%%%%%%%%%% 
\sectionfont{                     
    \LARGE % Font size
    \sectionrule{0pt}{0pt}{-8pt}{1pt} % Rule under Section name
}

\subsectionfont{
    \large % Font size
    \fontfamily{phv}\selectfont % Font family
    \sectionrule{0pt}{0pt}{-8pt}{1pt} % Rule under Subsection name
}

%%%%%%%%%%%%%%% Margins and Headers %%%%%%%%%%%%%%%
\geometry{
  a4paper,
  left=7mm,
  right=7mm,
  bottom=10mm,
  top=10mm
}

\pagestyle{empty} % Empty Headers

%%%%%%%%%%%%%%%%%%%%%%%%%%%%%%%%%%%%%%%% MACROS %%%%%%%%%%%%%%%%%%%%%%%%%%%%%%%%%%%%%%%%

%%%%%%%%%%%%%%% Link With an Icon %%%%%%%%%%%%%%% 
\newcommand{\link}[1]{
    \href{#1}{\faIcon{link}}
}

%%%%%%%%%%%%%%% Name Template %%%%%%%%%%%%%%% 
\newcommand{\name}[2]{
    % Name
    \Huge % Font size
    \raggedright \textbf{#1} \par

    \vspace*{0.3cm}
    
    % Profession
    \Large % Font size
    \raggedright #2 \par
    \normalsize \normalfont
}

%%%%%%%%%%%%%%% Contact Details %%%%%%%%%%%%%%%
\newcommand{\info}[2]{
    \faIcon{#2} \hspace{0.2em} #1
}

%%%%%%%%%%%%%%% Email %%%%%%%%%%%%%%%
\newcommand{\email}[1]{
    \info{#1}{envelope}
}

%%%%%%%%%%%%%%% Phone Number %%%%%%%%%%%%%%%
\newcommand{\phone}[1]{
    \info{#1}{mobile-alt}
}

%%%%%%%%%%%%%%% Address %%%%%%%%%%%%%%%
\newcommand{\address}[1]{
    \info{#1}{map-marker-alt}
}

%%%%%%%%%%%%%%% GitHub %%%%%%%%%%%%%%%
\newcommand{\github}[2]{
    \info{\href{#1}{\underline{#2}}}{github}
}

%%%%%%%%%%%%%%% LinkedIn %%%%%%%%%%%%%%%
\newcommand{\linkedin}[2]{
    \info{\href{#1}{\underline{#2}}}{linkedin}
}

%%%%%%%%%%%%%%% Website %%%%%%%%%%%%%%%
\newcommand{\website}[1]{
    \info{#1}{link}
}

%%%%%%%%%%%%%%% Draw Skill Bars %%%%%%%%%%%%%%% 
\newcommand{\drawskillbars}[1]{
    \begin{tikzpicture}
        % Draw 5 gray bars
        \foreach \i in {0, 1, 2, 3, 4}{
            \fill[lightgray] (\i * 0.7 + 0.2 *\i,0) rectangle (0.7 + \i * 0.7 + \i * 0.2,0.1);
        }
        
        % Draw number of black bars depending on the skill level
        \foreach \i in {#1}{
            \fill[black] (\i * 0.7 + 0.2 *\i,0) rectangle (0.7 + \i * 0.7 + \i * 0.2,0.1);
        }
    \end{tikzpicture} \par
}
    
%%%%%%%%%%%%%%% Skills %%%%%%%%%%%%%%%
\newcommand{\skill}[2]{
    % Name of the skill
    \large
    \noindent \hangafter=0
    \textmd{#1}
    \normalsize \par 
    % Skill bars
    \drawskillbars{#2}
    \vspace{1.5em}
}

%%%%%%%%%%%%%%% Language %%%%%%%%%%%%%%%
\newcommand{\lan}[2]{
    % Name of the language
    \large
    \noindent \hangafter=0
    \textmd{#1}
    % Knowledge level
    \drawskillbars{#2}
    \vspace{1em}
 }

%%%%%%%%%%%%%%% Education %%%%%%%%%%%%%%%
\newcommand{\education}[4]{
    % Name of the studies
    \noindent \large \parbox{.7\linewidth}{\textbf{#1}}
    % Duration in a Box
    \hfill \scriptsize
    \tcbox[enhanced,box align=base,nobeforeafter,colback=title,colframe=title,size=fbox,arc=0mm]{\textbf{#2}} \par
    \vspace{0.3em}
    % School Name 
    \large
    \noindent \color{title} \parbox{.7\linewidth}{\textsl{#3}} \par
    % Description
    \normalsize \color{black}
    \vspace*{0.3em}
    \small #4 
    \normalsize \par
}

%%%%%%%%%%%%%%% Work Experience %%%%%%%%%%%%%%%
\newcommand{\work}[4]{
    % Name of the Job
    \noindent \large \parbox{.7\linewidth}{\textbf{#1}}
    % Duration in a Box 
    \hfill \scriptsize
    \tcbox[enhanced,box align=base,nobeforeafter,colback=title,colframe=title,size=fbox,arc=0mm]{\textbf{#2}} \par
    \vspace{0.3em}
    % Name of the Employer
    \noindent \large \color{title} \parbox{.7\linewidth}{\textsl{#3}} \par
    % Description of the job
    \vspace*{0.3em} \color{black}
    \small #4 
    \normalsize \par
}

%%%%%%%%%%%%%%% Publications %%%%%%%%%%%%%%%
\newcommand{\pub}[5]{
    % Title
    \noindent \large \parbox{.7\linewidth}{\textbf{#1} \link{#5}}
    % Publication Date
    \hfill \scriptsize
    \tcbox[enhanced,box align=base,nobeforeafter,colback=title,colframe=title,size=fbox,arc=0mm]{\textbf{#2}} \par
    \vspace{0.3em}
    % Institution
    \large
    \noindent \color{title} \parbox{.7\linewidth}{\textsl{#3}} \par
    % Description
    \vspace*{0.3em} \color{black}
    \small \textit{#4} \par
    \normalsize \par 
}

\begin{document}
%%% TItle %%%
\tcbset{colframe=title,colback=title,arc=0mm}
\begin{tcolorbox}

    \begin{minipage}{0.3\textwidth} % Picture Area
        \includegraphics[width=0.8\textwidth]{photo.jpg} % Picture
    \end{minipage} \hfill
    \begin{minipage}{0.65\textwidth} % Name and Contact Info
        \name{David Legarre\\Saavedra}{Data Scientist} % Name and Profession
        \vspace{2em}
        \email{davidlegarre1@gmail.com} $\cdot$
        \phone{+34 684 21 33 79} \par \vspace{0.5em}
        %\info{july 7th 2000, Barcelona, Spain}{info-circle}
        \address{Barcelona, Spain}
    \end{minipage}

\end{tcolorbox}

%%% Sections %%%
\vspace*{-1em}
\tcbset{colframe=white,colback=white,arc=0mm, height=0.8\textheight}
\begin{tcolorbox}
    \vspace*{-0.5em}
    \begin{minipage}[t]{0.3\textwidth} % Side Panel (e.g. Skills, Links, Languages, etc.)
        \begin{tcolorbox}[height=0.8\textheight, grow to left by=0.6cm,colback=backdrop,colframe=backdrop,arc=0mm]
            
            \github{https://github.com/DavidLegarre}{DavidLegarre}\\
            \linkedin{https://www.linkedin.com/in/david-legarre-saavedra/}{David-Legarre-Saavedra}\\
            \kaggle{https://www.kaggle.com/davidlegarresaavedra}{DavidLegarreSaavedra}
            % Skills, the skill level is drawn as bars, input: skill name and an array starting from 0 and ending before 4
            \subsection*{Programming Languages and Tools}
            \skill{Python}{0, 1, 2, 3, 4}
            \skill{SQL}{0,1,2,3}
            \skill{R}{0,1,2}
            \skill{C}{0,1}
            \skill{Java}{0,1}

            \subsection*{Languages}
            \lan{Spanish}{0, 1, 2, 3, 4}
            \lan{Catalan}{0, 1, 2, 3, 4}
            \lan{English}{0, 1, 2, 3}
        \end{tcolorbox}
    \end{minipage}
    \begin{minipage}[t]{0.7\textwidth} % Main Panel (e.g. Education, Work Experience)
        \begin{tcolorbox}[grow to right by=0.75cm,height=0.8\textheight,colframe=white,colback=white]

            % Profile Section
            \section*{About Me}
            Recent graduate of a Bachelor's degree in Mathematical Engineering on 
            Data Science, I am a quick learner and flexible. 
            I am passionate about using data to drive insights and 
            solve complex problems, and I am looking for new opportunities to apply
            my skills and continue learning. 

            % Work Experience
            \section*{Work Experience}
            \work{Data Scientist}{Oct 2021 - Sept 2022}{Accenture}{
                Worked in several projects as a data scientist, performing Data Exploration and building predictive models
                \begin{itemize}
                    \item Conducted data analysis and modeling to support client projects.
                    Worked with MongoDB and GitLab.
                    \item Developed predictive 
                    models and algorithms using machine learning techniques such
                    as: linear regression, decision trees,
                    random forests and K-means.
                \end{itemize}
            }

            % Education
            \section*{Education}
            \education{BSc in Mathematical Engineering on Data Science}{Sep 2018 - Mar 2023}{Universitat Pompeu Fabra}{
                Skills:
                \begin{itemize}
                    \item Machine Learning \& Deep Learning: Pytorch and Scikit-learn
                    \item Statistical Methods for Data Science
                    \item Large-Scale Distributed Systems
                    \item Parallel \& Distributed Programming
                    \item Data Visualization: Power Bi and Tableau
                    \item Data Engineering: AWS, Hadoop, PySpark
                \end{itemize}
                
                Bachelor Thesis ``Deep Learning Methods for MRI coil Channel Optimization'' done in collaboration with 
                General Electric Helthcare and L'Hospital Clínic de Barcelona.
                A project in which Computer Vision and Deep Learning algorithms are used
                to improve the quaility by reducing the noise in MRI scans\\
            }
            \comeducation{English C1 Advanced- 190}{Jul 2018}{
                Cambridge University Assesment English
            }
        \end{tcolorbox}
    \end{minipage}
\end{tcolorbox}

\end{document}