\input{setup/preamble.tex}
\input{setup/macros.tex}

\begin{document}
%%% TItle %%%
\tcbset{colframe=title,colback=title,arc=0mm}
\begin{tcolorbox}

    \begin{minipage}{0.3\textwidth} % Picture Area
        \includegraphics[width=0.8\textwidth]{photo.jpg} % Picture
    \end{minipage} \hfill
    \begin{minipage}{0.65\textwidth} % Name and Contact Info
        \name{David Legarre\\Saavedra}{Data Scientist} % Name and Profession
        \vspace{2em}
        \email{davidlegarre1@gmail.com} $\cdot$
        \phone{+34 684 21 33 79} \par \vspace{0.5em}
        %\info{july 7th 2000, Barcelona, Spain}{info-circle}
        \address{Barcelona, Spain}
    \end{minipage}

\end{tcolorbox}

%%% Sections %%%
\vspace*{-1em}
\tcbset{colframe=white,colback=white,arc=0mm, height=0.8\textheight}
\begin{tcolorbox}
    \vspace*{-0.5em}
    \begin{minipage}[t]{0.3\textwidth} % Side Panel (e.g. Skills, Links, Languages, etc.)
        \begin{tcolorbox}[height=0.8\textheight, grow to left by=0.6cm,colback=backdrop,colframe=backdrop,arc=0mm]
            
            \github{https://github.com/DavidLegarre}{DavidLegarre}\\
            \linkedin{https://www.linkedin.com/in/david-legarre-saavedra/}{David-Legarre-Saavedra}\\
            \kaggle{https://www.kaggle.com/davidlegarresaavedra}{DavidLegarreSaavedra}
            % Skills, the skill level is drawn as bars, input: skill name and an array starting from 0 and ending before 4
            \subsection*{Programming Languages and Tools}
            \skill{Python}{0, 1, 2, 3, 4}
            \skill{SQL}{0,1,2,3}
            \skill{R}{0,1,2}
            \skill{C}{0,1}
            \skill{Java}{0,1}

            \subsection*{Languages}
            \lan{Spanish}{0, 1, 2, 3, 4}
            \lan{Catalan}{0, 1, 2, 3, 4}
            \lan{English}{0, 1, 2, 3}
        \end{tcolorbox}
    \end{minipage}
    \begin{minipage}[t]{0.7\textwidth} % Main Panel (e.g. Education, Work Experience)
        \begin{tcolorbox}[grow to right by=0.75cm,height=0.8\textheight,colframe=white,colback=white]

            % Profile Section
            \section*{About Me}
            Recent graduate of a Bachelor's degree in Mathematical Engineering on 
            Data Science, I am a quick learner and flexible. 
            I am passionate about using data to drive insights and 
            solve complex problems, and I am looking for new opportunities to apply
            my skills and continue learning. 

            % Work Experience
            \section*{Work Experience}
            \work{Data Scientist}{Oct 2021 - Sept 2022}{Accenture}{
                Worked in several projects as a data scientist, performing Data Exploration and building predictive models
                \begin{itemize}
                    \item Conducted data analysis and modeling to support client projects.
                    Worked with MongoDB and GitLab.
                    \item Developed predictive 
                    models and algorithms using machine learning techniques such
                    as: linear regression, decision trees,
                    random forests and K-means.
                \end{itemize}
            }

            % Education
            \section*{Education}
            \education{BSc in Mathematical Engineering on Data Science}{Sep 2018 - Mar 2023}{Universitat Pompeu Fabra}{
                Skills:
                \begin{itemize}
                    \item Machine Learning \& Deep Learning: Pytorch and Scikit-learn
                    \item Statistical Methods for Data Science
                    \item Large-Scale Distributed Systems
                    \item Parallel \& Distributed Programming
                    \item Data Visualization: Power Bi and Tableau
                    \item Data Engineering: AWS, Hadoop, PySpark
                \end{itemize}
                
                Bachelor Thesis ``Deep Learning Methods for MRI coil Channel Optimization'' done in collaboration with 
                General Electric Helthcare and L'Hospital Clínic de Barcelona.
                A project in which Computer Vision and Deep Learning algorithms are used
                to improve the quaility by reducing the noise in MRI scans\\
            }
            \comeducation{English C1 Advanced- 190}{Jul 2018}{
                Cambridge University Assesment English
            }
        \end{tcolorbox}
    \end{minipage}
\end{tcolorbox}

\end{document}